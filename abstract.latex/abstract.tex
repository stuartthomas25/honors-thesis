\documentclass[12pt]{report}
\usepackage{amsmath}
\usepackage{bbm}
\usepackage{amsfonts}

\setlength{\oddsidemargin}{0.5in}
\setlength{\evensidemargin}{0.5in}
\setlength{\textwidth}{6.0in}
\setlength{\topmargin}{.75in}
\setlength{\textheight}{8.6in}

% Modified by S. Aubin for the class of 2019 Physics theses

%
% This is a LaTeX template for an undergraduate honors or
%   senior thesis in physics at William & Mary. Created from
%   several theses from previous students of D.S. Armstrong,
%   pared down to bare bones.
%
%   you will need this present file (thesis_template.tex)
%   as well as the file  figure1.eps  to make the whole thing
%   (the latter is just an encapsulated postscript figure)
%     For simplicity, we don't use bibtex here; experts can choose
%   to use that more powerful way of dealing with citations.
%
%   To compile it, do:
% > pdflatex thesis_template.tex
% > pdflatex thesis_template.tex  (yes, you need to do it twice to
%                                  fill in the list of tables, figures
%                                  and references).



\begin{document}
\begin{abstract}
\setcounter{page}{5}
\addcontentsline{toc}{chapter}{Abstract}
\paragraph
\indent The $O(3)$ non-linear sigma model (NLSM) is a prototypical field theory for QCD and ferromagnetism, featuring topological qualities. Though the topological susceptibility $\chi_t$ should vanish in physical theories, lattice simulations of the NLSM find that $\chi_t$ diverges in the continuum limit. We study the effect of the gradient flow on this quantity using a Markov Chain Monte Carlo method, finding that a logarithmic divergence persists. This result supports a previous study and indicates that either the definition of topological charge is problematic or the NLSM has no well-defined continuum limit. We also introduce a $\theta$-term and analyze the topological charge as a function of $\theta$ under the gradient flow.

\end{abstract}
\end{document}



