% Please make sure you insert your
% data according to the instructions in PoSauthmanual.pdf
\documentclass[a4paper,11pt]{article}
\usepackage{pos}

\title{Topology of the $O(3)$ non-linear sigma model under the gradient flow}
%% \ShortTitle{Short Title for header}

\author*[a]{Stuart Thomas}
\author[a]{Christopher Monahan}

\affiliation[a]{Department of Physics, William \& Mary,  Williamsburg, Virginia 23187, USA}

%\affiliation[b]{Department, University,\\
%Street number, City, Country}

\emailAdd{snthomas01@email.wm.edu}
\emailAdd{cjmonahan@wm.edu}

\abstract{
The $O(3)$ non-linear sigma model (NLSM) is a prototypical field theory for QCD and ferromagnetism, featuring topological qualities. Though the topological susceptibility $\chi_t$ should vanish in physical theories, lattice simulations of the NLSM find that $\chi_t$ diverges in the continuum limit. We study the effect of the gradient flow on this quantity using a Markov Chain Monte Carlo method, finding that a logarithmic divergence persists. This result supports a previous study and indicates that either the definition of topological charge is problematic or the NLSM has no well-defined continuum limit. We also introduce a $\theta$-term and analyze the topological charge as a function of $\theta$ under the gradient flow.}

\FullConference{%
 The 38th International Symposium on Lattice Field Theory, LATTICE2021
  26th-30th July, 2021
  Zoom/Gather@Massachusetts Institute of Technology
}


\begin{document}
\maketitle


\section{The Non-Linear Sigma Model}
            We study the $O(3)$ non-linear sigma model (NLSM) in 1+1 dimensions, defined by the Euclidean action
            \begin{equation*}
                \label{eq:nlsm euclidean action}
                S_E = \frac{\beta}{2} \int d^2x \; \left[ \left(\partial_t \e\, \right)^2+ \left( \partial_x \e\,\right)^2 \right]
            \end{equation*}
        where, $\e$ is 3-component real vector constrained by $|\e\,|=1$ and $\beta$ is the inverse coupling constant. In solid-state systems, this model describes Heisenberg ferromagnets \cite{callan1985} and in nuclear physics, it acts as a prototype for quantum chromodynamics (QCD), the gauge theory that describes the strong nuclear force. In general, the NLSM shares key features with non-Abelian gauge theories such as QCD, including a mass gap and asymptotic freedom \cite{polyakov1975}. Therefore, the NLSM is a useful model for exploring the effect of these properties in a simpler system.


            %\item Prototypical model for strong nuclear force
                    %\item Models Heisenberg ferromagnets
                    %\item Applications to string theory
                %\end{itemize}
            %\item<4-> Merits
                %\begin{itemize}
                    %\item mass gap
                    %\item asymptotic freedom
                %\end{itemize}
        \end{itemize}

\begin{thebibliography}{99}
\bibitem{...}
....

\end{thebibliography}


\end{document}
