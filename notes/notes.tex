%        File: notes.tex
%     Created: Mon Jun 15 09:00 AM 2020 E
%


\documentclass[english]{article}
\usepackage[latin9]{inputenc}
\usepackage[letterpaper]{geometry}
\geometry{verbose}
\usepackage{textcomp}
\usepackage{amsmath}
\usepackage{graphicx}
\usepackage{siunitx}
\usepackage{subfigure}
\usepackage[english]{babel}
\usepackage{braket}
\usepackage{hyperref}
\newcommand{\bibliographytext}{References}

\begin{document}
\title{Research Notes}
\author{Stuart Thomas and Chris Monahan}
\date{Last updated \today}
\maketitle

\section{Monday, June 15}
\begin{enumerate}
    \item Preliminary meeting
    \item We will begin with $\phi^4$ term due to lower energy bound.
    \item Beginning with code in Python, switch to C/C++ if necessary.
\end{enumerate}

\section{Tuesday, June 16}
\begin{enumerate}
        
    \item Preliminary concepts to understand: 
        \begin{enumerate}
            \item scalar field theory on lattice
            \item Markov chains and Monte Carlo
            \item the gradient flow
            \item $O(n)$ symmetry
        \end{enumerate}

    \item General components to research, executed in parallel
        \begin{enumerate}    
            \item Reading
            \item Mathematical analysis
            \item Writing code
            \item Present (writing, plot generation,\dots)
        \end{enumerate}

    \item Things to get out of dissertation \cite{Schaich2006}
        \begin{enumerate}
            \item Markov chain
            \item Cluster algorithms

            \begin{enumerate}
                \item Metropolis is a method for narrowing possibilities by accepting only some changes. It can get stuck in a local mininum, loss of ergodicity. We solve these with cluster algoriths. Wolff grows clusters probabilistically and flips, while Swenson and Yang identifies clusters and flips them probabilistically.

                \item Near a phase transition, correlation length grows and changes become less likely to be accepted: need clusters. Clusters dont work far from the phase transition. This is manifested as a sequence of a few metropolis steps and a cluster step.
            \end{enumerate}
        \end{enumerate}
    \item Research plan
        \begin{enumerate}
            \item Start with 2D $\phi^4$

            \begin{enumerate}
                \item Set up lattice with sign flip for reflection
                \item Use Markov chain Monte Carlo to simulate.
                \item Measure, magnetization and suseptibility, Binder cumulant
            \end{enumerate}
            \item Transition to 3D, then maybe transition to C/C++.

            \item Implement the gradient flow

            \item move to 2-3d nonlinear sigma model. 

            \item Motivation: the nonlinear sigma model works for QCD given the asymtotic freedom. We may also want to explore topology.


        \end{enumerate}
\nocite{*}
\bibliographystyle{plain}
\bibliography{notes}

\end{enumerate}





\end{document}


