%        File: notes.tex
%     Created: Mon Jun 15 09:00 AM 2020 E
%


\documentclass[english]{article}
\usepackage[latin9]{inputenc}
\usepackage[letterpaper]{geometry}
\geometry{verbose}
\usepackage{textcomp}
\usepackage{amsmath}
\usepackage{graphicx}
\usepackage{siunitx}
\usepackage{subfigure}
\usepackage[english]{babel}
\usepackage{braket}
\usepackage{hyperref}
\newcommand{\bibliographytext}{References}

\begin{document}
\title{Research Notes}
\author{Stuart Thomas and Chris Monahan}
\date{Last updated \today}
\maketitle

\section{Monday, June 15}
\begin{enumerate}
    \item Preliminary meeting
    \item We will begin with $\phi^4$ term due to lower energy bound.
    \item Beginning with code in Python, switch to C/C++ if necessary.
\end{enumerate}

\section{Tuesday, June 16}
\begin{enumerate}
        
    \item Preliminary concepts to understand: 
    \begin{enumerate}
        \item scalar field theory on lattice
        \item Markov chains and Monte Carlo
        \item the gradient flow
        \item $O(n)$ symmetry
    \end{enumerate}

    \item General components to research, executed in parallel
    \begin{enumerate}    
        \item Reading
        \item Mathematical analysis
        \item Writing code
        \item Present (writing, plot generation,\dots)
    \end{enumerate}

    \item Things to get out of dissertation \cite{Schaich2006}
    \begin{enumerate}
        \item Markov chain
        \item Cluster algorithms

        \begin{enumerate}
            \item Metropolis is a method for narrowing possibilities by accepting only some changes. It can get stuck in a local mininum, loss of ergodicity. We solve these with cluster algoriths. Wolff grows clusters probabilistically and flips, while Swenson and Yang identifies clusters and flips them probabilistically.

            \item Near a phase transition, correlation length grows and changes become less likely to be accepted: need clusters. Clusters dont work far from the phase transition. This is manifested as a sequence of a few metropolis steps and a cluster step.
        \end{enumerate}
    \end{enumerate}
    \item Research plan
    \begin{enumerate}
        \item Start with 2D $\phi^4$
        \begin{enumerate}
            \item Set up lattice with sign flip for reflection
            \item Use Markov chain Monte Carlo to simulate.
            \item Measure, magnetization and suseptibility, Binder cumulant
        \end{enumerate}
        \item Transition to 3D, then maybe transition to C/C++.
        \item Implement the gradient flow
        \item move to 2-3d nonlinear sigma model. 
        \item Motivation: the nonlinear sigma model works for QCD given the asymtotic freedom. We may also want to explore topology.


    \end{enumerate}
\end{enumerate}


\section{Wednesday, June 17}
\begin{enumerate}
    \item Code tips:
    \begin{enumerate}
        \item Try Swendsen-Wang algorithm in addition to Wolff
        \item Print out time taken
        \item Optimize Hamiltonian
        \item Save every tenth measurement or store configurations to calculate path integral. Exclude thermaliztion (first 200)
        \item Write in terms of sweeps, not iterations
        \item Parallelize (look into checkerboard algorithm)
        \item Implement Binder cumulant and suseptibility
        \item Store every few states
        \item Look into multigrid algorithm

    \end{enumerate}

    \item Reading on Monte Carlo Markov chain and cluster algorithms.
\end{enumerate}

\section{Friday, June 19}
    \begin{enumerate}
        \item Looking over code
        \begin{enumerate}
            \item Might be too slow, move to C/C++ eventually?
            \item Why is the energy increasing with metropolis algorithm?
            \item Shift to using action instead of Hamiltonian.
            \item Transition from Broken Phase, look at $\mu_0^2$ term.
            \item Profile code for possible optimizations. % http://users.physik.fu-berlin.de/~kleinert/kleiner_reb8/psfiles/phi4.pdf
        \end{enumerate}
    \end{enumerate}

 \section{Monday, June 22}
    \begin{enumerate}
        \item Coding
        \begin{enumerate}
            \item Add plots to \texttt{.gitignore}.
            \item Try parallelizing code, using either multigrid or checkerboard algorithm
        \end{enumerate}

        \item Reading
        \begin{enumerate}
            \item Start to focus more on understanding the theory behind research.
            \item Read dissertation \cite{Schaich2006} Chap. 6.5 and 6.6, take notes on questions.
            \item Newman \cite{Newman1999} (Main textbook for Monte Carlo in Statistical Physics)
        \end{enumerate}

        \item Just some things to remember
        \begin{enumerate}
            \item Correlation functions correlate values in statistical systems and relate to propagators in QFT.
            \item The problem of renormalization: $a\rightarrow0$ leads to unbounded correlation function. As real physical lengths are measured in terms of the lattice constant, these sometimes tend to infinity.
        \end{enumerate}
    \end{enumerate}


 \section{Wednesday, June 24}
    \begin{enumerate}

        \item Code
        \begin{enumerate}
            \item Parallelize
            \item Transition to numpy
        \end{enumerate}
        \item Theory Question
        \begin{enumerate}
            \item renormalize and regularlize: what do they mean?
            \item Look at LePage
        \end{enumerate}

    \end{enumerate}

 \section{Tuesday, June 30}
    \begin{enumerate}
        \item Code
        \begin{enumerate}
            \item Try \texttt{mpi4py}.
            \item Move to 3D (this may decrease parallel overhead)
        \end{enumerate}

        \item Reading
        \begin{enumerate}
            \item Continue Reading Collins, others. 
        \end{enumerate}
    \end{enumerate}

 \section{Thursday, July 2}
    \begin{enumerate}
        \item Coding
            \begin{enumerate}
                \item Continue implementing MPI
                \item Parallelization may be more apparent in 3D
            \end{enumerate}
        \item Reading
            \begin{enumerate }
                \item Dirac fermions, represented by 4D spinor field (spin up/down, electron/positron)    
                \item Look at Tong (Chap. 4)
                \item Charge (See Tong, Noether's Theorem)
            \end{enumerate}



        \item Next week back: start gradient flow on linear phi4 model 

            
    \end{enumerate}

\nocite{*}
\bibliographystyle{plain}
\bibliography{notes}
\end{document}
