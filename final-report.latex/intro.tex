Quantum field theory (QFT) is an framework used to describe a range of physical phenomena to a remarkable degree of accuracy. Paired with the Standard Model, QFT provides the prevailing basis for all small-scale physics (that is, where general relativity does not apply) and is the fundamental tool for studying particle physics. QFT also plays a critical role in condensed matter physics through effective field theories which model emergent phenomena such as phonons and quasiparticles. Compared to experiment, QFT is remarkably accurate, famously predicting the electron $g$-factor to eleven significant figures \cite{odom2006}, arguably the most accurate prediction in all of science.

However this power comes at a cost: the study of quantum fields is rife with infinities. A na\"ive treatment of quantum field theory produces divergent values for physical quantities, a clearly impossible result. Since the 1950s, this issue has been resolved for a large number of models--- most notably quantum electrodynamics--- through perturbation theory and the renormalization group. This counter-intuitive technique exploits the freedom of parameters such as mass and electric charge. Since these constants cannot be directly measured, renormalization allows them to become infinite, thereby cancelling the infinities in physical results. Unfortunately, not all theories are perturbatively renormalizable. 

One such example is the \textit{non-linear sigma model} (NLSM), a prototypical theory in both condensed matter and particle physics. In solid-state systems, this model describes Heisenberg ferromagnets \cite{callan1985} and in nuclear physics, it acts as a prototype for quantum chromodynamics (QCD), the gauge theory that describes the strong nuclear force. In general, the NLSM shares key features with non-Abelian gauge theories such as QCD, including a mass gap and asymptotic freedom \cite{polyakov1975}. Therefore, the NLSM is a useful model for exploring the effect of these properties using a simpler system.

In this study, we specifically consider the $O(3)$ non-linear sigma model in $1+1$ dimensions (one dimension of space, one dimension of time). This theory exhibits topological properties such as \textit{instantons}, or classical field solutions at local minima of the action in Euclidean space. Solutions such as these are ``topologically protected,'' meaning they cannot evolve into the vacuum state via small fluctuations. Therefore, it is critically important to take topology into account when considering quantum field theories in a cosmology and high energy physics \cite{goddard1986}. Additionally, topological stability may become a key tool for fault-tolerant quantum computers \cite{kitaev1997}. In these devices, topology could protect the delicate quantum states necessary for information processing.

Since the NLSM in not perturbatively renormalizable, we require nonperturbative techniques to study topological effects. An alternative solution is to place the field on a discretized Euclidean lattice, a technique originally used for quantum chromodynamics \cite{wilson1974}. After this transformation, fields become computationally calculable. This process introduces the lattice spacing as a length scale $a$ where the \textit{continuum limit} is defined as taking $a$ to zero. We expect physical observables to converge in the continuum limit, however this is not always the case. As an example, states of definite angular momentum mix when discretized on a rectangular lattice due to a breaking of continuous rotational symmetry. This causes some operators that depend on angular momentum to suffer divergences.

In this study, we focus on one observable that diverges in the continuum: the topological susceptibility. The topological susceptibility in the 1+1 $O(3)$ NLSM has been the subject of debate for the past four decades \cite{bietenholz2018} since it is unclear if a convergent solution exists. While the some analytical arguments argue the topological susceptibility should approach zero in the continuum limit, numerical results predict infinities. In QCD, mathematical techniques proved that the susceptibility vanishes in the continuum limit \cite{giusti2004}, a fact supported by numerical calculation \cite{giusti2004}.

To remedy this issue, we consider the gradient flow, a technique designed to remove divergences in operators. By dampening high-frequency fluctuations, the gradient flow reduces terms which scale with the inverse lattice spacing, making some observables finite on the lattice \cite{monahan2016}. In QCD, the gradient flow successfully makes the topological susceptibility finite in numerical calculations \cite{bruno2014}, corroborating the analytical result in \cite{giusti2004}. This success has motivated the usage of the gradient flow to calculate the topological susceptibility in the 1+1 $O(3)$ NLSM, though the observable still diverges in the continuum limit \cite{bietenholz2018}. 

A second perspective on the topological susceptibility arises from the introduction of a $\theta$-term into the field Lagrangian. This term drives the vacuum state into a topological phase \cite{allessalom2008}. Differentiating the field's partition function with respect to $\theta$ yields a value proportional to the topological susceptibility. The effect of nonzero $\theta$ on the theory therefore should reflect the divergence in the continuum limit.

In this work we verify the divergence of the topological susceptibility and develop a clearer picture of how the $\theta$-term affects the topology of the 1+1 $O(3)$ NLSM.


\section{Method Overview}

To numerically study the topological qualities of the non-linear sigma model, we first implement a Markov chain Monte Carlo simulation. We initially construct a proof-of-concept Python program that models the simpler $\phi^4$ model (see Sec.~\ref{sec:phi4}). After comparing with existing literature, we transition to a C++ simulation for efficiency, implementing the non-linear sigma model in larger lattices. Since the gradient flow has no exact solution in the non-linear sigma model, we implement a numerical solution using a fourth-order Runge-Kutta approximation with automatic step sized. By applying the gradient flow to every configuration in the sample, we can measure its effect on the topological charge and susceptibility.

\section{Summer Research}

A portion of this work began during the summer of 2021 using funding from the 1693 Scholars Program. This preliminary research consisted of literature review and numerical tests with an Ising model simulation as well as an initial implementation of the $\phi^4$ model. The $\phi^4$ calculation performed in this study and the entirety of the NLSM portion occurred during the academic year as part of the PHYS 495/496 Honors course.


\section{Conventions}
\begin{itemize}
    \item Throughout this paper, we use natural units, i.e. $\hbar = 1$ and $c=1$.
    \item We use Einstein summation notation, an implicit sum over repeated spacetime indices. For example, if $x^\mu$ is a spacetime four-vector and $x_\mu$ is its covariant form, the term
        \begin{align*}
            x^\mu x_\mu &= \sum^4_{\mu=0} x^\mu x_\mu \\
            &= x_0^2-x_1^2-x_2^2-x_3^2.
        \end{align*}
\end{itemize}

