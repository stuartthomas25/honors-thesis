Using a Monte Carlo simulation, we have analyzed two main quantum field theories in 1+1 spacetime dimensions: the $\phi^4$ model and the $O(3)$ non-linear sigma model. We used the path integral formulation of QFT to simulate these quantum fields as statistical systems on a Euclidean lattice and extract information about the phase transitions and topology. To perform these calculations, we run a series of Metropolis sweeps interspersed with Wolff cluster steps, creating a Markov chain of samples. We ensure that each measurement is effectively independent by measuring the autocorrelation and thermalize the lattice to keep all measurements near the action minima.

In the $\phi^4$ model, we verified a phase transition at $m_0^2=-0.7$ using the magnitude of the average magnetization, the magnetic susceptibility, the Binder cumulant and the bimodality. This process verified the effectiveness of the Monte Carlo computational system. We then generalized our system to the $O(3)$ non-linear sigma model and transitioned to a C++ code base. We confirmed our calculations with known results by measuring the internal energy and magnetic susceptibility. We calculated the topological susceptibility $\chi_t$ to analyze its divergence in the continuum limit, confirming that this is indeed the case even at finite flow time. Specifically, the topological susceptibility under the gradient flow follows either a power law relationship or a logarithmic relationship as a function of lattice size, both of which diverge as $L\rightarrow\infty$.  At flow time $\tau=5t_0$, the $\chi^2/DOF$ goodness of fit values are of \powchi and \logchi respectively. 

Finally, we demonstrated the relationship between the topological charge and the $\theta$ in the topological action. This plot demonstrates quantitatively the different rates of divergence in the topological susceptibility. In the $\tau=0$ case, we see the slope rapidly approach positive infinity at $\theta=0$. While this transition occurs more slowly with the application of the gradient flow, the slope continues to increase.

Though the logarithmic and power-law functions visually fit the susceptibility data well, the $\chi^2$ values are high. This imprecision can be attributed to underestimated errors. Since the $\tau=0$ case features a more acceptable fit ($\chi^2/DOF=1.8$ for the power-law), the gradient flow seemingly contributes to this error. While the Jackknife method accurately estimates statistical errors of the sample, we did not incorporate any systematic errors arising from the gradient flow calculation. Future work could reduce the tolerance of the adaptive step size algorithm to make this calculation more accurate, though this change increases computational requirements substantially. Furthermore, a larger number of lattice sizes may provide a more complete picture of the continuum limit.

Procedurally, we found that the MPI parallelization and checkerboard algorithm were unnecessary in this calculation. The computation time of the Runge-Kutta algorithm for computing the gradient flow far outweighed that of the Monte Carlo simulation. A possible improvement could be a parallelization of the Runge-Kutta algorithm or a more accurate approximation to the gradient flow. Additionally, the performance of the Python simulation was slower by up to two orders of magnitude. Future studies should therefore rely solely on an efficient programming language for Monte Carlo techniques.

In the context of the long-standing question of topological suscpetibility in the NLSM, this study has diminished the plausibility that ultraviolet fluctuations cause the divergence. Instead, we are left to consider the other two options outlined in \cite{berg1981}: that the definition of the topological charge density is problematic or that the NLSM does not have a decent continuum limit. Future work could include similar numerical calculations using a different definition of the topological charge.

Generally, this study has implications in both condensed matter and nuclear physics. Though the calculation of the $\chi_t$ divergence confirms existing literature, the relationship between the topological charge and $\theta$ in the flow time was previously unexplored. Beyond the convergence of $\chi_t$, this relationship has applications in condensed matter where different values of $\theta$ can describe spin-chains of either fermions or bosons \cite{bogli2012}. Furthermore, the mass gap has a strong relationship with the $\theta$-term in the topological NLSM, featuring massive and massless regimes \cite{allessalom2008}.


